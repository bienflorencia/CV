%%%%%%%%%%%%%%%%%%%%%%%%%%%%%%%%%%%%%%%%%%%%%%%%%%%%%%%%%%%%%%%%%%%%%%%%
%%%%%%%%%%%%%%%%%%%%%% Simple LaTeX CV Template %%%%%%%%%%%%%%%%%%%%%%%%
%%%%%%%%%%%%%%%%%%%%%%%%%%%%%%%%%%%%%%%%%%%%%%%%%%%%%%%%%%%%%%%%%%%%%%%%

%%%%%%%%%%%%%%%%%%%%%%%%%%%%%%%%%%%%%%%%%%%%%%%%%%%%%%%%%%%%%%%%%%%%%%%%
%% NOTE: If you find that it says                                     %%
%%                                                                    %%
%%                           1 of ??                                  %%
%%                                                                    %%
%% at the bottom of your first page, this means that the AUX file     %%
%% was not available when you ran LaTeX on this source. Simply RERUN  %%
%% LaTeX to get the ``??'' replaced with the number of the last page  %%
%% of the document. The AUX file will be generated on the first run   %%
%% of LaTeX and used on the second run to fill in all of the          %%
%% references.                                                        %%
%%%%%%%%%%%%%%%%%%%%%%%%%%%%%%%%%%%%%%%%%%%%%%%%%%%%%%%%%%%%%%%%%%%%%%%%

%%%%%%%%%%%%%%%%%%%%%%%%%%%% Document Setup %%%%%%%%%%%%%%%%%%%%%%%%%%%%

% Don't like 10pt? Try 11pt or 12pt
\documentclass[10pt]{article}

% The automated optical recognition software used to digitize resume
% information works best with fonts that do not have serifs. This
% command uses a sans serif font throughout. Uncomment both lines (or at
% least the second) to restore a Roman font (i.e., a font with serifs).
%\usepackage{times}
%\renewcommand{\familydefault}{\sfdefault}

% This is a helpful package that puts math inside length specifications
\usepackage{calc}

 % reverse order of lists
\usepackage{etaremune}
\usepackage{graphicx}
\usepackage{wrapfig}
\usepackage{multicol}

%\renewcommand{\familydefault}{\sfdefault}
%\usepackage[sfdefault]{FiraSans} %% option 'sfdefault' activates Fira Sans as the default text font
%\usepackage[T1]{fontenc}
%\renewcommand*\oldstylenums[1]{{\firaoldstyle #1}}

\usepackage[utf8]{inputenc}

% add thin horizontal lines
\newcommand{\HRule}{\rule{\linewidth}{0.05mm}}


% Layout: Puts the section titles on left side of page
\reversemarginpar

%
%         PAPER SIZE, PAGE NUMBER, AND DOCUMENT LAYOUT NOTES:
%
% The next \usepackage line changes the layout for CV style section
% headings as marginal notes. It also sets up the paper size as either
% letter or A4. By default, letter was used. If A4 paper is desired,
% comment out the letterpaper lines and uncomment the a4paper lines.
%
% As you can see, the margin widths and section title widths can be
% easily adjusted.
%
% ALSO: Notice that the includefoot option can be commented OUT in order
% to put the PAGE NUMBER *IN* the bottom margin. This will make the
% effective text area larger.
%
% IF YOU WISH TO REMOVE THE ``of LASTPAGE'' next to each page number,
% see the note about the +LP and -LP lines below. Comment out the +LP
% and uncomment the -LP.
%
% IF YOU WISH TO REMOVE PAGE NUMBERS, be sure that the includefoot line
% is uncommented and ALSO uncomment the \pagestyle{empty} a few lines
% below.
%

%% Use these lines for letter-sized paper
\usepackage[paper=letterpaper,
            %includefoot, % Uncomment to put page number above margin
            marginparwidth=1.2in,     % Length of section titles
            marginparsep=.05in,       % Space between titles and text
            margin=1in,               % 1 inch margins
            includemp]{geometry}

%% Use these lines for A4-sized paper
%\usepackage[paper=a4paper,
%            %includefoot, % Uncomment to put page number above margin
%            marginparwidth=30.5mm,    % Length of section titles
%            marginparsep=1.5mm,       % Space between titles and text
%            margin=25mm,              % 25mm margins
%            includemp]{geometry}

%% More layout: Get rid of indenting throughout entire document
\setlength{\parindent}{0in}

\usepackage[shortlabels]{enumitem}

% Simpler bibsections for CV sections
% (thanks to natbib for inspiration)
%
% * For lists of references with hanging indents and no numbers:
%
%   \begin{bibsection}
%       \item ...
%   \end{bibsection}
%
% * For numbered lists of references (with hanging indents):
%
%   \begin{bibenum}
%       \item ...
%   \end{bibenum}
%
%   Note that bibenum numbers continuously throughout. To reset the
%   counter, use
%
%   \restartlist{bibenum}
%
%   at the place where you want the numbering to reset.

\makeatletter
\newlength{\bibhang}
\setlength{\bibhang}{1em}
\newlength{\bibsep}
 {\@listi \global\bibsep\itemsep \global\advance\bibsep by\parsep}
\newlist{bibsection}{itemize}{3}
\setlist[bibsection]{label=,leftmargin=\bibhang,%
        itemindent=-\bibhang,
        itemsep=\bibsep,parsep=\z@,partopsep=0pt,
        topsep=0pt}
\newlist{bibenum}{enumerate}{3}
\setlist[bibenum]{label=[\arabic*],resume,leftmargin={\bibhang+\widthof{[999]}},%
        itemindent=-\bibhang,
        itemsep=\bibsep,parsep=\z@,partopsep=0pt,
        topsep=0pt}
\let\oldendbibenum\endbibenum
\def\endbibenum{\oldendbibenum\vspace{-.6\baselineskip}}
\let\oldendbibsection\endbibsection
\def\endbibsection{\oldendbibsection\vspace{-.6\baselineskip}}
\makeatother

%% Reference the last page in the page number
%
% NOTE: comment the +LP line and uncomment the -LP line to have page
%       numbers without the ``of ##'' last page reference)
%
% NOTE: uncomment the \pagestyle{empty} line to get rid of all page
%       numbers (make sure includefoot is commented out above)
%
\usepackage{fancyhdr,lastpage}
\pagestyle{fancy}
%\pagestyle{empty}      % Uncomment this to get rid of page numbers
\fancyhf{}\renewcommand{\headrulewidth}{0pt}
%\renewcommand{\headheight}{16pt}
\fancyhead[R]{\small{\textit{Florencia Grattarola  (Curriculumn Vitae, page
      \thepage~of~\protect\pageref*{LastPage}) }}}


% Finally, give us PDF bookmarks
\usepackage{color,hyperref}
\definecolor{darkblue}{rgb}{0.243,0.243,0.7}
\hypersetup{colorlinks,breaklinks,
            linkcolor=darkblue,urlcolor=darkblue,
            anchorcolor=darkblue,citecolor=darkblue}

%%%%%%%%%%%%%%%%%%%%%%%% End Document Setup %%%%%%%%%%%%%%%%%%%%%%%%%%%%


%%%%%%%%%%%%%%%%%%%%%%%%%%% Helper Commands %%%%%%%%%%%%%%%%%%%%%%%%%%%%

%%% HEADING AT TOP OF CURRICULUM VITAE

% The title (name) with a horizontal rule under it
% (optional argument typesets an object right-justified across from name
%  as well)
%
% Usage: \makeheading{name}
%        OR
%        \makeheading[right_object]{name}
%
% Place at top of document. It should be the first thing.
% If ``right_object'' is provided in the square-braced optional
% argument, it will be right justified on the same line as ``name'' at
% the top of the CV. For example:
%
%       \makeheading[\emph{Curriculum vitae}]{Your Name}
%
% will put an emphasized ``Curriculum vitae'' at the top of the document
% as a title. Likewise, a picture could be included:
%
%   \makeheading[\includegraphics[height=1.5in]{my_picutre}]{Your Name}
%
% the picture will be flush right across from the name.
\newcommand{\makeheading}[2][]%
        {\hspace*{-\marginparsep minus \marginparwidth}%
         \begin{minipage}[t]{\textwidth+\marginparwidth+\marginparsep}%
             {\large \bfseries #2 \hfill #1}\\[-0.15\baselineskip]%
                 \rule{\columnwidth}{1pt}%
         \end{minipage}}

%%% SECTION HEADINGS

% The section headings. Flush left in small caps down pseudo-margin.
%
% Usage: \section{section name}
\renewcommand{\section}[1]{\pagebreak[3]%
    \vspace{1.3\baselineskip}%
    \phantomsection\addcontentsline{toc}{section}{#1}%
    \noindent\llap{\scshape\smash{\parbox[t]{\marginparwidth}{\hyphenpenalty=10000\raggedright #1}}}%
    \vspace{-\baselineskip}\par}

%%% LISTS

% This macro alters a list by removing some of the space that follows the list
% (is used by lists below)
\newcommand*\fixendlist[1]{%
    \expandafter\let\csname preFixEndListend#1\expandafter\endcsname\csname end#1\endcsname
    \expandafter\def\csname end#1\endcsname{\csname preFixEndListend#1\endcsname\vspace{-0.6\baselineskip}}}

% These macros help ensure that items in outer-type lists do not get
% separated from the next line by a page break
% (they are used by lists below)
\let\originalItem\item
\newcommand*\fixouterlist[1]{%
    \expandafter\let\csname preFixOuterList#1\expandafter\endcsname\csname #1\endcsname
    \expandafter\def\csname #1\endcsname{\csname preFixOuterList#1\endcsname\let\oldItem\item\def\item{\pagebreak[2]\oldItem}}
    \expandafter\let\csname preFixOuterListend#1\expandafter\endcsname\csname end#1\endcsname
    \expandafter\def\csname end#1\endcsname{\let\item\oldItem\csname preFixOuterListend#1\endcsname}}
\newcommand*\fixinnerlist[1]{%
    \expandafter\let\csname preFixInnerList#1\expandafter\endcsname\csname #1\endcsname
    \expandafter\def\csname #1\endcsname{\let\oldItem\item\let\item\originalItem\csname preFixInnerList#1\endcsname}
    \expandafter\let\csname preFixInnerListend#1\expandafter\endcsname\csname end#1\endcsname
    \expandafter\def\csname end#1\endcsname{\csname preFixInnerListend#1\endcsname\let\item\oldItem}}

% An itemize-style list with lots of space between items
%
% Usage:
%   \begin{outerlist}
%       \item ...    % (or \item[] for no bullet)
%   \end{outerlist}
\newlist{outerlist}{itemize}{3}
    \setlist[outerlist]{label=\enskip\textbullet,leftmargin=*}
    \fixendlist{outerlist}
    \fixouterlist{outerlist}

% An environment IDENTICAL to outerlist that has better pre-list spacing
% when used as the first thing in a \section
%
% Usage:
%   \begin{lonelist}
%       \item ...    % (or \item[] for no bullet)
%   \end{lonelist}
\newlist{lonelist}{itemize}{3}
    \setlist[lonelist]{label=\enskip\textbullet,leftmargin=*,partopsep=0pt,topsep=0pt}
    \fixendlist{lonelist}
    \fixouterlist{lonelist}

% An itemize-style list with little space between items
%
% Usage:
%   \begin{innerlist}
%       \item ...    % (or \item[] for no bullet)
%   \end{innerlist}
\newlist{innerlist}{itemize}{3}
\renewcommand\labelitemi{}
\setlist[innerlist]{label=\textbullet,leftmargin=*,parsep=0pt,itemsep=0pt,topsep=0pt,partopsep=0pt}
\fixinnerlist{innerlist}

% An environment IDENTICAL to innerlist that has better pre-list spacing
% when used as the first thing in a \section
%
% Usage:
%   \begin{loneinnerlist}
%       \item ...    % (or \item[] for no bullet)
%   \end{loneinnerlist}
\newlist{loneinnerlist}{itemize}{3}
    \setlist[loneinnerlist]{label=\enskip\textbullet,leftmargin=*,parsep=0pt,itemsep=0pt,topsep=0pt,partopsep=0pt}
    \fixendlist{loneinnerlist}
    \fixinnerlist{loneinnerlist}

%%% EXTRA SPACE

% To add some paragraph space between lines.
% This also tells LaTeX to preferably break a page on one of these gaps
% if there is a needed pagebreak nearby.
\newcommand{\blankline}{\quad\pagebreak[3]}
\newcommand{\halfblankline}{\quad\vspace{-0.5\baselineskip}\pagebreak[3]}

%%% FORMATTING MACROS

% Uses hyperref to link DOI
\newcommand\doilink[1]{\href{http://dx.doi.org/#1}{#1}}
\newcommand\doi[1]{doi:\doilink{#1}}

% For \url{SOME_URL}, links SOME_URL to the url SOME_URL
\providecommand*\url[1]{\href{#1}{#1}}
% Same as above, but pretty-prints SOME_URL in teletype fixed-width font
\renewcommand*\url[1]{\href{#1}{\texttt{#1}}}

% For \email{ADDRESS}, links ADDRESS to the url mailto:ADDRESS
\providecommand*\email[1]{\href{mailto:#1}{#1}}
% Same as above, but pretty-prints ADDRESS in teletype fixed-width font
%\renewcommand*\email[1]{\href{mailto:#1}{\texttt{#1}}}

%\providecommand\BibTeX{{\rm B\kern-.05em{\sc i\kern-.025em b}\kern-.08em
%    T\kern-.1667em\lower.7ex\hbox{E}\kern-.125emX}}
%\providecommand\BibTeX{{\rm B\kern-.05em{\sc i\kern-.025em b}\kern-.08em
%    \TeX}}
\providecommand\BibTeX{{B\kern-.05em{\sc i\kern-.025em b}\kern-.08em
    \TeX}}
\providecommand\Matlab{\textsc{Matlab}}

% Custom hyphenation rules for words that LaTeX has trouble with
\hyphenation{bio-mim-ic-ry bio-in-spi-ra-tion re-us-a-ble pro-vid-er}

%%%%%%%%%%%%%%%%%%%%%%%% End Helper Commands %%%%%%%%%%%%%%%%%%%%%%%%%%%

\usepackage{wasysym}

%%%%%%%%%%%%%%%%%%%%%%%%% Begin CV Document %%%%%%%%%%%%%%%%%%%%%%%%%%%%

\begin{document}

\makeheading{Florencia Grattarola -- Curriculum Vitae}

\thispagestyle{empty}

\section{Contact information}

	Department of Spatial Sciences \\
	Czech University of Life Sciences in Prague \\
	Kamycka 129, 16500 Praha - Suchdol\\
    	E-mail: \email{grattarola@fzp.czu.cz}\\
    	Web: \href{https://flograttarola.com}{flograttarola.com}\\
    	Phone: +420 774 975 962‬

\HRule

\section{Education}

\begin{innerlist}

\item[]{\bf  Ph.D., Life Sciences}
\hfill {2017--2021} \\
Evolution and Ecology Research Group, School of Life Sciences, {\bf University of Lincoln}, United Kingdom. Dissertation: \textit{Biodiversity informatics and macroecological patterns of biodiversity: Uruguay as a model region}. Advisor: \href{http://selectiondynamics.weebly.com/daniel-pincheira-donoso.html}{Daniel Pincheira-Donoso}.\\

\item[]{\bf  M.Sc., Biological Sciences (Genetics)}
\hfill {2013--2015} \\
Department of Biodiversity and Genetics, Clemente Estable Biological Research Institute. School of Science, {\bf Universidad de la Rep\'{u}blica}, Montevideo, Uruguay. Dissertation: \textit{Contributions of molecular ecology
to the study of mammals in Uruguay}. Advisors: \href{https://scholar.google.com.uy/citations?user=FSrtqaMAAAAJ&hl=en}{Susana González} and \href{https://scholar.google.com/citations?user=uWh1ONQAAAAJ&hl=es}{Mariana Cosse}.\\

\item[]{\bf  B.Sc. (Hons), Biological Sciences}
\hfill {2006--2012} \\
School of Science, {\bf Universidad de la Rep\'{u}blica}, Montevideo, Uruguay. Advisor: \href{https://scholar.google.com/citations?user=uWh1ONQAAAAJ&hl=es}{Mariana Cosse}.\\

\end{innerlist}

\HRule

\section{Professional experience}

\begin{innerlist}

\item[]{\bf Postdoc}
\hfill {2021--present} \\
Department of Spatial Sciences, {\bf Czech University of Life Sciences}, Prague, Czech Republic.\\

\item[]{\bf Intern}
\hfill {2021} \\
Knowledge Management, {\bf UN Environmental Programme - World Conservation Monitoring Centre}, Cambridge, UK.\\

\item[]{\bf Demonstrator}
\hfill {2018--2020} \\
School of Life Sciences, {\bf University of Lincoln}, Lincoln, UK.\\

\item[]{\bf Research assistant}
\hfill {2017} \\
Department of Animal Genetics, School of Veterinary, {\bf Universidad de la Rep\'{u}blica}, Montevideo, Uruguay.\\

\item[]{\bf Project Manager}
\hfill {2017} \\
Wildlife Uruguay NGO, Montevideo, Uruguay.\\

\item[]{\bf Field Assistant}
\hfill {2017} \\
Project Rural Futures, {\bf Technische Universität Berlin}, Uruguay.\\

\item[]{\bf Lab Technician}
\hfill {2015--2016}\\
Department of Genetics, {\bf Rural Association of Uruguay}, Montevideo, Uruguay.\\

\item[]{\bf Research Assistant}
\hfill {2015--2016}\\
Institute of Hygiene, School of Medicine, {\bf Universidad de la Rep\'{u}blica}, Montevideo, Uruguay. Bioinformatics of Salmonella\\

\item[]{\bf Research Assistant}
\hfill {2012--2015}\\
Department of Biodiversity and Genetics, {\bf Clemente Estable Biological Research Institute}, Montevideo, Uruguay.\\

\item[]{\bf Teaching Assistant}
\hfill {2012--2014}\\
Extension Unit, School of Science, {\bf Universidad de la Rep\'{u}blica}, Montevideo, Uruguay.\\\\\\

\end{innerlist}


\HRule

\section{Peer-reviewed articles}

\begin{etaremune}

\item Pincheira-Donoso D., Harvey L.P., \underline{Grattarola F.}, Winter M., Jara M., Cotter S.C., Tregenza T. \& Hodgson D.J. (2021) The multiple origins of sexual size dimorphism in global amphibians. \textit{\textbf{Global Ecology and Biogeography}}, 30(2): 443-458. [\href{https://doi.org/10.1111/geb.13230}{link}]

\item \underline{Grattarola F.}, Martínez-Lanfranco, J.A., ... \& Pincheira-Donoso D. (2020) Multiple forms of hotspots of tetrapod biodiversity and the challenges of open-access data scarcity. \textit{\textbf{Scientific Reports}}, 10(1): 1-15. [\href{https://doi.org/10.1038/s41598-020-79074-8}{link}]

\item \underline{Grattarola F.}, Mai, P., ... \& Pincheira-Donoso D. (2020) Biodiversidata: A novel dataset for the vascular plant species diversity in Uruguay. \textit{\textbf{Biodiversity Data Journal}}, 8: e56850. [\href{https://doi.org/10.3897/BDJ.8.e56850}{link}]

\item \underline{Grattarola F.} \& Rodríguez-Tricot L. (2020) Mammals of Paso Centurión, an area with relicts of Atlantic Forest in Uruguay. \textit{\textbf{Neotropical Biology and Conservation}}, 15(3): 267–283. [\href{https://doi.org/10.3897/neotropical.15.e53062}{link}]

\item D’Alessandro B., Pérez Escanda V., Balestrazzi V., \underline{Grattarola F.}, ... \& Betancor L.  (2020) Comparative genomics of Salmonella enterica serovar Enteritidis ST-11 isolated in Uruguay reveals lineages associated with particular epidemiological traits. \textit{\textbf{Scientific Reports}}, 10: 3638. [\href{https://doi.org/10.1038/s41598-020-60502-8}{link}]

\item \underline{Grattarola F.}, Botto, G., ... \& Pincheira-Donoso D. (2019) Biodiversidata: an open-access biodiversity database for Uruguay. \textit{\textbf{Biodiversity Data Journal}}, 7: e36226. [\href{https://doi.org/10.3897/BDJ.7.e36226}{link}]

\item \underline{Grattarola F.} \& Pincheira-Donoso D. (2019) Data-sharing in Uruguay, the vision of collectors and data users. \textit{\textbf{Boletín de la Sociedad Zoológica del Uruguay}}, 28(1): 1-14. [\href{https://doi.org/10.26462/28.1.1}{link}]

\item Bergós L., \underline{Grattarola F.}, Barreneche J.M., Herández D., \& González S. (2018) Fogones de Fauna: an experience of participatory monitoring of wildlife in rural Uruguay. \textit{\textbf{Society \& Animals}}, 26(2), 171-185. [\href{https://doi.org/10.1163/15685306-12341497}{link}]

\item Chouhy M., Santos C., Gaucher L., \underline{Grattarola F.}, ... \& Perazza, G. (2017) At the frontiers of knowledge: the quest for an integral training course on society-nature. \textit{\textbf{Integralidad Sobre Ruedas}}, 4(1): 62–77. [\href{https://ojs.fhce.edu.uy/index.php/insoru/article/view/234}{link}]

\item Cosse M., \underline{Grattarola F.}  \& Mannise N. (2017) A novel real-time TaqMan\texttrademark PCR assay for simultaneous detection of Neotropical fox species using noninvasive samples based on cytochrome C oxidase subunit II. \textit{\textbf{Mammal Research}}, 62(4): 405-411. [\href{https://doi.org/10.1007/s13364-017-0328-y}{link}]

\item \underline{Grattarola F.}, Hernández D., ... \& Rodríguez-Tricot R. (2016) First record of jaguarundi (Puma yagouaroundi) (Mammalia: Carnivora: Felidae) in Uruguay,with comments about  participatory monitoring. \textit{\textbf{Boletín de la Sociedad Zoológica del Uruguay}}, 25(1): 85-91. [\href{http://szu.org.uy/journal/index.php/Bol_SZU/article/view/23}{link}]

\item \underline{Grattarola F.}, Cosse M. \& González S. (2015) A novel primer set for mammal species identification from feces samples. \textit{\textbf{Conservation Genetic Resources}}, 7: 57–59. [\href{https://doi.org/10.1007/s12686-014-0359-5}{link}]

\end{etaremune}

\HRule

\section{Service for Journals}

{\bf Reviewer for}: \textit{Biodiversity Data Journal, Global Ecology and Biogeography, Journal of Biogeography}

\HRule

%\newpage
\section{Citation \\ Metrics}

Accessed 11 Mar 2022:

\begin{itemize}
\item \textbf{Google Scholar:} 67 citations, h-index  = 5.

\end{itemize}

\HRule

\section{Students}

\begin{innerlist}

\item[]{\bf Enzo Cavalli}, BSc (Honours) in Biological Sciences, School of Science, {\bf Universidad de la Rep\'{u}blica} (Advisor, finished in 2019)

\end{innerlist}

\HRule

\section{Lecturing}

\begin{innerlist}

\item[]{\bf Modern Mammals} - Universidad de la Rep\'{u}blica, Uruguay
\hfill {2017-2019}\\
Lecture in the graduate course of Palaeontology, for 15-20 students. Invited by Dr. Richard Fari\~{n}a. 
[\href{https://flograttarola.com/pdf/mamiferos_julana.pdf}{link to slides}]

\medskip

\item[]{\bf Management and analysis of camera trap records using R software} - Ministry of Environment, Uruguay
\hfill {2018}\\
Course for NGOs, researchers and government people. Taught together with Juan M. Barreneche and Gabriel Perazza.
[\href{https://github.com/bienflorencia/curso_camtrapR}{link to syllabus}]

\medskip

\item[]{\bf Introduction to Python} - Universidad de la Rep\'{u}blica, Uruguay
\hfill {2016--2017}\\
Lecture at posgraduate course of Introduction to command line and programming for bioinformatic analysis, for 25-30 students in Biological Sciences. Invited by Dr. Andr\'{i}s Iriarte. 
[\href{https://github.com/bienflorencia/clases_python}{link to syllabus}]

\medskip

\item[]{\bf Society \& Nature} - Universidad de la Rep\'{u}blica, Uruguay
\hfill {2016--2017}\\
Graduate course for ca. 20 students of Anthropology, Biology, Geography and Social Sciences. Taught together with Carlos Santos, Javier Taks, Magdalena Chouhy, Andrea Garay, Luc\'{i}a Berg\'{o}s and Gabriel Perazza.
[\href{https://udelar.edu.uy/retema/actividades/grupos-de-trabajo/}{link to S\&N Group}]

\end{innerlist}

\HRule

\section{Grants, funding}

\begin{innerlist}

\item[]{\bf NatGeo Explorer}
\hfill {2021--2022} \\
1-year grant, 'NaturalistaUY: the iNaturalista community in Uruguay'. National Geographic Society, United States.
\\

\item[]{\bf Ph.D. International Fellowship}
\hfill {2017--2020} \\
3-year grant at the Univeristy of Lincoln, Lincoln, UK. National Agency for Research and Innovation (ANII) Uruguay POS\_EXT\_2016\_1\_136663
\\

\item[]{\bf Post-graduate National Fellowship}
\hfill {2013--2015} \\
2-year grant at the Universidad de la Rep\'{u}blica, Montevideo, Uruguay. National Agency for Research and Innovation (ANII) Uruguay POS\_NAC\_2012\_1\_8976
\\

\item[]{\bf Internship for Post-graduate Students}
\hfill {2015} \\
3-month at the Universidad Federal de Minas Gerais, Belo Horizonte, Brazil. Association of Universities of the Montevideo Group (AUGM) Uruguay
\\

\item[]{\bf Internship for Post-graduate Students}
\hfill {2013} \\
1-month at the Mississippi State University, Starkville, United States. PEDECIBA, Uruguay
\\

\item[]{\bf Undergraduate Fellowship.}
\hfill {2010--2011} \\
1-year grant at the Universidad de la Rep\'{u}blica, Montevideo, Uruguay. National Agency for Research and Innovation (ANII) Uruguay BE\_INI\_2010\_1961
\\

\end{innerlist}

\HRule

\section{Skills}

{\bf Software}: R, Python, Windows, Linux, Bash, ArcGIS, QGIS, JAGS, HTML, CSS, SQL, Markdown, Hugo-R, jekyll, WordPress, GBIF IPT, DarwinCore standard, InkScape and Adobe suites.

\medskip

{\bf Lab}: DNA extraction, PCR, RT-PCR, Taqman probe, analysis of mitochondrial, nuclear and hypervariable markers (microsatellites and SNPs), population genetics, landscape genetics and phylogenetics, NGS sequencing (Illumina and IonTorrent), alignment, annotation, comparative genomics.

\medskip

{\bf Field}: non-invasive wildlife monitoring, camera-traps, bat sensor EchoMeter, AudioMoth acoustic logger, mammal track identification.

\medskip

{\bf Statistics}: GLM, hierarchical (mixed-effects) models, machine learning, Bayesian inference, data visualization.

\medskip

{\bf Languages}: English; Portuguese (excellent); Spanish (native).

\HRule

\section{References}


Dr. \textbf{Petr Keil}, Czech University of Life Sciences, Czech Republic. \phone +420 776 371 840. \email{keil@fzp.czu.cz}

\medskip

Dr. \textbf{Daniel Pincheira-Donoso}, Queen’s University Belfast, UK. \phone +44(0) 28 9097 2967. \email{d.pincheira-donoso@qub.ac.uk}
\medskip

Dr. \textbf{Lauren Weatherdon}, UN Environment Programme World Conservation Monitoring Centre, UK. \phone +44(0) 12 2327 7314. \email{lauren.weatherdon@unep-wcmc.org}

\medskip

Prof. \textbf{Susana González}, Clemente Estable Biological Research Institute, Uruguay. \phone +598 24861417.
\email{sugonza9@yahoo.com}



\end{document}\grid
\grid
