%%%%%%%%%%%%%%%%%%%%%%%%%%%%%%%%%%%%%%%%%
% Developer CV
% LaTeX Template
% Version 1.0 (28/1/19)
%
% This template originates from:
% http://www.LaTeXTemplates.com
%
% Authors:
% Jan Vorisek (jan@vorisek.me)
% Based on a template by Jan Küster (info@jankuester.com)
% Modified for LaTeX Templates by Vel (vel@LaTeXTemplates.com)
%
% License:
% The MIT License (see included LICENSE file)
%
%%%%%%%%%%%%%%%%%%%%%%%%%%%%%%%%%%%%%%%%%

%----------------------------------------------------------------------------------------
%	PACKAGES AND OTHER DOCUMENT CONFIGURATIONS
%----------------------------------------------------------------------------------------

\documentclass[9pt]{developercv} % Default font size, values from 8-12pt are recommended

%----------------------------------------------------------------------------------------

\begin{document}

%----------------------------------------------------------------------------------------
%	TITLE AND CONTACT INFORMATION

%----------------------------------------------------------------------------------------

\begin{minipage}[t]{0.45\textwidth} % 45% of the page width for name
	\vspace{-\baselineskip} % Required for vertically aligning minipages
	
	% If your name is very short, use just one of the lines below
	% If your name is very long, reduce the font size or make the minipage wider and reduce the others proportionately
	\colorbox{black}{{\HUGE\textcolor{white}{\textbf{\MakeUppercase{Florencia}}}}} % First name
	
	\colorbox{black}{{\HUGE\textcolor{white}{\textbf{\MakeUppercase{Grattarola}}}}} % Last name
	
	\vspace{6pt}
	
	{\huge Postdoc Researcher} % Career or current job title
\end{minipage}
\begin{minipage}[t]{0.275\textwidth} % 27.5% of the page width for the first row of icons
	\vspace{-\baselineskip} % Required for vertically aligning minipages
	
	% The first parameter is the FontAwesome icon name, the second is the box size and the third is the text
	% Other icons can be found by referring to fontawesome.pdf (supplied with the template) and using the word after \fa in the command for the icon you want
	\icon{MapMarker}{12}{Prague, Czech Republic}\\
	\icon{Phone}{12}{+420 774 975 962}\\
	\icon{At}{12}{\href{mailto:flograttarola@gmail.com}{flograttarola@gmail.com}}\\	
	\icon{Globe}{12}{\href{https://flograttarola.com}{flograttarola.com}}\\
\end{minipage}
\begin{minipage}[t]{0.275\textwidth} % 27.5% of the page width for the second row of icons
	\vspace{-\baselineskip} % Required for vertically aligning minipages
	
	% The first parameter is the FontAwesome icon name, the second is the box size and the third is the text
	% Other icons can be found by referring to fontawesome.pdf (supplied with the template) and using the word after \fa in the command for the icon you want
	\icon{Github}{12}{\href{https://github.com/bienflorencia}{github.com/bienflorencia}}\\
	\icon{Orcid}{12}{\href{https://orcid.org/0000-0001-8282-5732}{0000-0001-8282-5732}}\\
	\icon{Twitter}{12}{\href{https://twitter.com/flograttarola}{@flograttarola}}\\
	\icon{Mastodon}{12}{\href{https://ecoevo.social/@flograttarola}{@flograttarola}}\\
%	\icon{Telegram}{12}{\href{https://t.me/bienflorencia}{@bienflorencia}}\\
\end{minipage}

\vspace{0.5cm}

%----------------------------------------------------------------------------------------
%	INTRODUCTION, SKILLS AND TECHNOLOGIES
%----------------------------------------------------------------------------------------

%\cvsect{Bio}
%
%\begin{minipage}[t]{0.5\textwidth} % 40% of the page width for the introduction text
%	\vspace{-\baselineskip} % Required for vertically aligning minipages
%	
%	I am a Uruguayan biologist (PhD) doing research in macroecology and biodiversity informatics. I am a postdoc researcher at the Czech University of Life Sciences at the \href{https://petrkeil.github.io/}{\underline{MOBI Lab}}, my current work focuses on the study of spatial and temporal patterns of biodiversity. I am also the founder of \href{https://biodiversidata.org/}{\underline{Biodiversidata}} (Uruguayan Consortium of Biodiversity Data), a member of the \href{https://julana.org/}{\underline{JULANA NGO}}, the admin for \href{https://www.naturalista.uy/}{\underline{NaturalistaUY}} (the iNaturalist site for Uruguay), and a \href{https://explorers.nationalgeographic.org/directory/florencia-grattarola}{\underline{NatGeoExplorer}}.\\ % Dummy text
%\end{minipage}
%\hfill % Whitespace between
%\begin{minipage}[t]{0.4\textwidth} % 50% of the page for the skills bar chart
%	\vspace{-\baselineskip} % Required for vertically aligning minipages
%	\begin{barchart}{5.5}
%		\baritem{Open Science}{50}
%		\baritem{Biodiversity Informatics}{50}
%		\baritem{Macroecology}{50}
%		\baritem{GIS}{50}
%		\baritem{Genomics}{50}
%		\baritem{Community Science}{50}
%	\end{barchart}
%\end{minipage}

%\begin{center}
%	\bubbles{5/Eclipse, 6/git, 4/Office, 3/Inkscape, 3/Blender}
%\end{center}

\cvsect{Bio}

I am a Uruguayan biologist (PhD) doing research in Macroecology and Biodiversity Informatics. I currently work as a postdoc researcher at the Czech University of Life Sciences (\href{https://petrkeil.github.io/}{\underline{MOBI Lab}}), focused on studying spatial and temporal patterns of biodiversity. I am also the founder of \href{https://biodiversidata.org/}{\underline{Biodiversidata}} (Uruguayan Consortium of Biodiversity Data), a member of the \href{https://julana.org/}{\underline{JULANA NGO}}, the administrator of \href{https://www.naturalista.uy/}{\underline{NaturalistaUY}} (the iNaturalist site for Uruguay), and a \href{https://www.nationalgeographic.org/find-explorers/ED0031R000029s0gjQAA}{\underline{NatGeoExplorer}}. I co-chair the RDA Sharing Rewards and Credit Interest Group  (\href{https://www.rd-alliance.org/groups/sharing-rewards-and-credit-sharc-ig}{\underline{SHARC}}). I am a GBIF \href{https://www.gbif.org/article/1ye7qAa9Z2HVSn85bfSmLP/who-are-the-gbif-biodiversity-open-data-ambassadors}{\underline{Open Data Ambassador}} and I am also a \href{https://carpentries.org}{\underline{The Carpentries}} Instructor. \\\

%\begin{minipage}[t]{0.5\textwidth} % 50% of the page for the skills bar chart
%	\vspace{-\baselineskip} % Required for vertically aligning minipages
%	\begin{barchart}{10}
%		\baritem{Programming, GIS and Data Science}{0}
%		\baritem{Bioinformatics}{0}
%		\baritem{Participative Monitoring}{0}
%		\baritem{Molecular Analysis}{0}
%		\baritem{Field Work}{0}
%		\baritem{Web development and Graphic design}{0}
%	\end{barchart}
%\end{minipage}
%\hfill % Whitespace between
%\begin{minipage}[t]{0.5\textwidth} % 50% of the page for the skills bar chart
%	\vspace{-\baselineskip} % Required for vertically aligning minipages
%	\begin{barchart}{0}
%		\baritem{Programming, GIS and Data Science}{0}
%		\baritem{Bioinformatics}{0}
%		\baritem{Participative Monitoring}{0}
%		\baritem{Molecular Analysis}{0}
%		\baritem{Field Work}{0}
%		\baritem{Web development and Graphic design}{0}
%	\end{barchart}
%\end{minipage}

%----------------------------------------------------------------------------------------
%	EDUCATION
%----------------------------------------------------------------------------------------

\cvsect{Education}

\begin{entrylist}
	\entry
		{2017 -- 2021}
		{Ph.D., Life Sciences}
		{University of Lincoln, United Kingdom}
		{Evolution and Ecology Research Group, School of Life Sciences. Dissertation: Biodiversity informatics and macroecological patterns of biodiversity: Uruguay as a model region. Advisor: \href{http://selectiondynamics.weebly.com/daniel-pincheira-donoso.html}{Daniel Pincheira-Donoso}}
	\entry
		{2013 -- 2015}
		{M.Sc., Biological Sciences (Genetics)}
		{Universidad de la Rep\'{u}blica, Uruguay}
		{Department of Biodiversity and Genetics, Clemente Estable Biological Research Institute. Dissertation: Contributions of molecular ecology to the study of mammals in Uruguay. Advisors: \href{https://scholar.google.com.uy/citations?user=FSrtqaMAAAAJ&hl=en}{Susana González} and \href{https://scholar.google.com/citations?user=uWh1ONQAAAAJ&hl=es}{Mariana Cosse}}
	\entry
		{2006 -- 2012}
		{B.Sc. (Hons), Biological Sciences}
		{Universidad de la Rep\'{u}blica, Uruguay}
		{Department of Biodiversity and Genetics, Clemente Estable Biological Research Institute. Project: Development of molecular tools for the taxonomic identification of different carnivore species. Advisor: \href{https://scholar.google.com/citations?user=uWh1ONQAAAAJ&hl=es}{Mariana Cosse}}
\end{entrylist}

%----------------------------------------------------------------------------------------
%	EXPERIENCE
%----------------------------------------------------------------------------------------

\cvsect{Professional Experience}

\begin{entrylist}
%	\entry
%		{2015 -- 2018\\\footnotesize{part time}}
%		{Full stack developer}
%		{Famous Eshop Inc.}
%		{\lorem\lorem\\ \texttt{PHP}\slashsep\texttt{JS}\slashsep\texttt{MariaDB}\slashsep\texttt{Linux}}
	\entry
		{11/2023 -- present}
		{Postdoc}
		{Czech University of Life Sciences Prague (Czech Republic)}
		{Project 'BEAST: Biodiversity dynamics across a continuum of space, time, and their scales'. MOBI Lab, ERC Funding}
	\entry
		{05/2023 -- 08/2023}
		{Consultant}
		{Agencia Nacional de Investigación e Innovación (Uruguay)}
		{Open Research Data Incentive Programme}
	\entry
		{04/2021 -- 10/2023}
		{Postdoc}
		{Czech University of Life Sciences Prague (Czech Republic)}
		{Project 'Range dynamics of Neotropical carnivores'. MOBI Lab, REES ČZU Funding}
	\entry
		{01/2021 -- 04/2021}
		{Intern}
		{UN Environmental Programme - World Conservation Monitoring Centre (Cambridge, UK)}
		{Knowledge Management internship in the Conserving Land and Seascapes Team}
	\entry
		{08/2018 -- 12/2020\\\footnotesize{part time}}
		{Demonstrator}
		{University of Lincoln (Lincoln, United Kingdom)}
		{Courses of Zoology, Paleontology, and Biological Anthropology at the School of Life Sciences}
	\entry
		{01/2017 -- 07/2017\\\footnotesize{part time}}
		{Project Manager}
		{Vida Silvestre NGO (Montevideo, Uruguay)}
		{Project 'Conservation in private protected areas of Uruguay'}
	\entry
		{11/2016 -- 03/2017}
		{Research Assistant}
		{Universidad de la Rep\'{u}blica (Montevideo, Uruguay)}
		{Bioinformatics and genomics at the Department of Animal Genetics, School of Veterinary}
	\entry
		{10/2016 -- 11/2016}
		{Field Assistant}
		{Technische Universität Berlin (Uruguay)}
		{Sampling plots of vascular plants for the project 'Rural Futures'}
	\entry
		{11/2015 -- 10/2016\\\footnotesize{part time}}
		{Technician}
		{Rural Association of Uruguay (Montevideo, Uruguay)}
		{Livestock pedigree data management}
	\entry
		{11/2015 -- 10/2016\\\footnotesize{part time}}
		{Research Assistant}
		{Universidad de la Rep\'{u}blica (Montevideo, Uruguay)}
		{Project 'Bioinformatics and comparative genomics of Salmonella' at the Institute of Hygiene, School of Medicine}
	\entry
		{01/2012 -- 12/2015}
		{Research Assistant}
		{Clemente Estable Biological Research Institute (Montevideo, Uruguay)}
		{Research and teaching at the Department of Biodiversity and Genetics}
	\entry
		{05/2012 -- 04/2014\\\footnotesize{part time}}
		{Teaching Assistant}
		{Universidad de la Rep\'{u}blica (Montevideo, Uruguay)}
		{Organization of outreach ativities at the Extension Unit, School of Science}
\end{entrylist}

%----------------------------------------------------------------------------------------
%	PUBLICATIONS
%----------------------------------------------------------------------------------------

\cvsect{Peer-reviewed articles}

\begin{etaremune}

\item Kallivalappil R., \underline{Grattarola F.}, de Alwis Pitts D., Cotter S.C., \& Pincheira-Donoso D. (2024) Species diversity and extinction risk of vertebrate pollinators in India. \textit{\textbf{Biodiversity and Conservation}}. [\href{https://doi.org/10.1007/s10531-024-02848-3}{doi}]

\item \underline{Grattarola F.}, Rodríguez-Tricot L., Zarucki M. \& Laufer G. (2024) Status of the invasion of Carpobrotus edulis in Uruguay based on citizen science records. \textit{\textbf{Biological Invasions}}. [\href{https://doi.org/10.1007/s10530-023-03242-w}{doi}]

\item Tschernosterová K., Trávníčková E., \underline{Grattarola F.} \& Keil P. (2023) SPARSE 1.0: a template for databases of species inventories, with an open example of Czech birds. \textit{\textbf{Biodiversity Data Journal}}, 11: e108731. [\href{https://doi.org/10.3897/BDJ.11.e108731}{doi}]

\item Finn C., \underline{Grattarola F.}  \& Pincheira-Donoso D. (2023) More losers than winners: investigating Anthropocene defaunation through the diversity of population trends. \textit{\textbf{Biological Reviews}}. [\href{https://doi.org/10.1111/brv.12974}{doi}]

\item \underline{Grattarola F.}, Bowler D.E. \& Keil P. (2023) Integrating presence-only and presence–absence data to model changes in species geographic ranges: An example in the Neotropics. \textit{\textbf{Journal of Biogeography}}. [\href{https://doi.org/10.1111/jbi.14622}{doi}]

\item Moudrý V., Keil P., Cord A.F., Gábor L., Lecours V., Zarzo-Arias A., Barták V.,  Malavasi M., Rocchini D., Torresani M., Gdulová K. , \underline{Grattarola F.}, Leroy F., Marchetto E., Prošek J., Thouverai E., Wild J.  \& Šímová P. (2023) Scale mismatches between predictor and response variables in species distribution modelling: a review of practices for appropriate grain selection. \textit{\textbf{Progress in Physical Geography}}. [\href{https://doi.org/10.1177/03091333231156362}{doi}]

\item Zarzo-Arias A., Penteriani V.,  Gábor L., Šímová P.,  \underline{Grattarola F.}, \& Moudrý V. (2022) Importance of Data Selection and Filtering in Species Distribution Models: A Case Study on the Cantabrian Brown Bear. \textit{\textbf{Ecosphere}}, 13(12): e4284. [\href{https://doi.org/10.1002/ecs2.4284}{doi}]

\item Callaghan C.T., Mesaglio T., Ascher J.S, Brooks T.M., Cabras A.A., Chandler M., Cornwell W.K., Ríos-Málaver I.C., Dankowicz E., Dhiya’ulhaq N.U., Fuller R.A., Galindo-Leal C., \underline{Grattarola F.}, [...], \& Young A.N. (2022) The Benefits of Contributing to the Citizen Science Platform INaturalist as an Identifier. \textit{\textbf{PLOS Biology}}, 20(11): e3001843. [\href{https://doi.org/10.1371/journal.pbio.3001843}{doi}]

\item Pincheira-Donoso D., Harvey L.P., \underline{Grattarola F.}, Jara M., Cotter S.C., Tregenza T. \& Hodgson D.J. (2021) The multiple origins of sexual size dimorphism in global amphibians. \textit{\textbf{Global Ecology and Biogeography}}, 30(2): 443-458. [\href{https://doi.org/10.1111/geb.13230}{doi}]

\item \underline{Grattarola F.}, Martínez-Lanfranco, J.A., [...], \& Pincheira-Donoso D. (2020) Multiple forms of hotspots of tetrapod biodiversity and the challenges of open-access data scarcity. \textit{\textbf{Scientific Reports}}, 10(1): 1-15. [\href{https://doi.org/10.1038/s41598-020-79074-8}{doi}]

\item \underline{Grattarola F.}, Mai, P., [...], \& Pincheira-Donoso D. (2020) Biodiversidata: A novel dataset for the vascular plant species diversity in Uruguay. \textit{\textbf{Biodiversity Data Journal}}, 8: e56850. [\href{https://doi.org/10.3897/BDJ.8.e56850}{doi}]

\item \underline{Grattarola F.} \& Rodríguez-Tricot L. (2020) Mammals of Paso Centurión, an area with relicts of Atlantic Forest in Uruguay. \textit{\textbf{Neotropical Biology and Conservation}}, 15(3): 267–283. [\href{https://doi.org/10.3897/neotropical.15.e53062}{doi}]

\item D’Alessandro B., Pérez Escanda V., Balestrazzi V., \underline{Grattarola F.}, [...], \& Betancor L.  (2020) Comparative genomics of Salmonella enterica serovar Enteritidis ST-11 isolated in Uruguay reveals lineages associated with particular epidemiological traits. \textit{\textbf{Scientific Reports}}, 10: 3638. [\href{https://doi.org/10.1038/s41598-020-60502-8}{doi}]

\item \underline{Grattarola F.}, Botto, G., [...], \& Pincheira-Donoso D. (2019) Biodiversidata: an open-access biodiversity database for Uruguay. \textit{\textbf{Biodiversity Data Journal}}, 7: e36226. [\href{https://doi.org/10.3897/BDJ.7.e36226}{doi}]

\item \underline{Grattarola F.} \& Pincheira-Donoso D. (2019) Data-sharing in Uruguay, the vision of collectors and data users. \textit{\textbf{Boletín de la Sociedad Zoológica del Uruguay}}, 28(1): 1-14. [\href{https://doi.org/10.26462/28.1.1}{doi}]

\item Bergós L., \underline{Grattarola F.}, Barreneche J.M., Herández D., \& González S. (2018) Fogones de Fauna: an experience of participatory monitoring of wildlife in rural Uruguay. \textit{\textbf{Society \& Animals}}, 26(2), 171-185. [\href{https://doi.org/10.1163/15685306-12341497}{doi}]

\item Chouhy M., Santos C., Gaucher L., \underline{Grattarola F.}, [...], \& Perazza, G. (2017) At the frontiers of knowledge: the quest for an integral training course on society-nature. \textit{\textbf{Integralidad Sobre Ruedas}}, 4(1): 62–77. [\href{https://ojs.fhce.edu.uy/index.php/insoru/article/view/234}{link}]

\item Cosse M., \underline{Grattarola F.}  \& Mannise N. (2017) A novel real-time TaqMan\texttrademark PCR assay for simultaneous detection of Neotropical fox species using noninvasive samples based on cytochrome C oxidase subunit II. \textit{\textbf{Mammal Research}}, 62(4): 405-411. [\href{https://doi.org/10.1007/s13364-017-0328-y}{doi}]

\item \underline{Grattarola F.}, Hernández D., [...], \& Rodríguez-Tricot R. (2016) First record of jaguarundi (Puma yagouaroundi) (Mammalia: Carnivora: Felidae) in Uruguay,with comments about  participatory monitoring. \textit{\textbf{Boletín de la Sociedad Zoológica del Uruguay}}, 25(1): 85-91. [\href{http://szu.org.uy/journal/index.php/Bol_SZU/article/view/23}{link}]

\item \underline{Grattarola F.}, Cosse M. \& González S. (2015) A novel primer set for mammal species identification from feces samples. \textit{\textbf{Conservation Genetic Resources}}, 7: 57–59. [\href{https://doi.org/10.1007/s12686-014-0359-5}{doi}]

\end{etaremune}


\cvsect{Pre-print articles}

\begin{etaremune}

\item \underline{Grattarola F.}, Bergós L., Carabio M., Montiel R. \& González S. (2024) NaturalistaUY in Uruguay: a case of community science in Latin America from a critical perspective. \textit{\textbf{LatArXiv}}. [\href{https://doi.org/10.62059/LatArXiv.preprints.180}{doi}]

\item \underline{Grattarola F.}, Tschernosterová K. \& Keil P. (2024) A continental-wide decline of occupancy and diversity in five charismatic Neotropical carnivores. \textit{\textbf{SSRN}}. [\href{https://doi.org/10.2139/ssrn.4916634}{doi}]

\item \underline{Grattarola F.}, Shmagun H., Erdmann C., Cambon-Thomsen A., Thomsen M., Kim J. \& Mabile L. (2024) Gaps between Open Science activities and actual recognition systems: Insights from an international survey. \textit{\textbf{SocArXiv}}. [\href{https://doi.org/10.31235/osf.io/hru2x}{doi}]

\item \underline{Grattarola F.}, Laufer G., Rodríguez-Tricot L., González E. \& Teixeira de Mello F. (2024) Desafíos y barreras para la disponibilización de datos abiertos de biodiversidad en Uruguay. \textit{\textbf{EcoEvoRxiv}}. [\href{https://doi.org/10.32942/X2RK6K}{doi}]

\item \underline{Grattarola F.}, Rodríguez-Tricot L., Zarucki M. \& Laufer G. (2023) Status of the invasion of Carpobrotus edulis in Uruguay based on community science records. \textit{\textbf{Research Square PREPRINT}}. [\href{https://doi.org/10.21203/rs.3.rs-3185397/v1}{doi}]

\item Tschernosterová K., Trávníčková E., \underline{Grattarola F.} \& Keil P. (2023) SPARSE 1.0: a template for databases of species inventories, with an open example of Czech birds. \textit{\textbf{ARPHA Preprints}}. [\href{https://doi.org/10.3897/arphapreprints.e110098}{doi}]

\item \underline{Grattarola F.}, Bowler D.E. \& Keil P. (2022) Integrating presence-only and presence-absence data to model changes in species geographic ranges: An example of yaguarundí in Latin America. \textit{\textbf{EcoEvoRxiv}}. [\href{https://doi.org/10.32942/osf.io/67c4u}{doi}]

\item Pincheira-Donoso D., Harvey L.P., Guirguis J., Goodyear L.E.B., Finn C., Johnson J.V., \underline{Grattarola F.} (2022) Temporal and spatial patterns of vertebrate extinctions during the Anthropocene. \textit{\textbf{bioRxiv}}. [\href{https://doi.org/10.1101/2022.05.05.490605}{doi}]

\end{etaremune}

\cvsect{Books}

\begin{etaremune}

\item Chouhy M., Bergós L., Garay A., \underline{Grattarola F.}, Perazza G., Santos C. \& Taks J. (2022) Relaciones Sociedad-Naturaleza en Paso Centurión. Aportes desde una trayectoria integral universitaria en la frontera noreste de Uruguay. \textit{\textbf{Servicio Central de Extensión y Actividades en el Medio Universidad de la República}}. [\href{https://udelar.edu.uy/retema/wp-content/uploads/sites/30/2022/09/relaciones_sociedad-naturaleza_en_paso_centurion-comprimido-1.pdf}{pdf}]

\end{etaremune}

\hfill 
%----------------------------------------------------------------------------------------
%	GRANTS
%----------------------------------------------------------------------------------------

\cvsect{Grants, Funding}

\begin{entrylist}
	\entry
		{2022}
		{Capacity Enhancement Support Programme}
		{Global Biodiversity Information Facility (GBIF)}
		{\textsterling 14,000 | Project 'Extending knowledge on biodiversity data quality and publication in the Spanish-speaking community' CESP2021-007 [\href{https://www.gbif.org/project/CESP2021-007/}{link}]}
	\entry
		{2021 -- 2022}
		{Citizen Science for Species Discovery Grant}
		{National Geographic Society}
		{\textsterling 25,000 | Project 'Naturalista UY – The iNaturalist community for Uruguay' (Uruguay)}
	\entry
		{2019}
		{College of Science Grant for Postgraduate Students}
		{University of Lincoln }
		{\textsterling 750 | Attend the conference ‘Biodiversity Next’ (Leiden, The Netherlands)}
	\entry
		{2019}
		{Travel Grant for Postgraduate Students}
		{University of Lincoln }
		{\textsterling 500 | Attend the course ‘Advancing in statistical modelling for evolutionary biologists and ecologists using R’ (Glasgow, Scotland)}
	\entry
		{2018}
		{Mobility Fund for Postgraduate Research Students}
		{Santander - University of Lincoln}
		{\textsterling 500 | Deliver the ‘First meeting of the members of Biodiversidata’ (Montevideo, Uruguay)}
	\entry
		{2018}
		{Travel Grant for Early Career European Researchers \& Scientists }
		{Research Data Alliance (RDA)}
		{\textsterling 1,500 | Attend the '11th RDA Plenary Meeting’ (Berlin, Germany)}
	\entry
		{2017 -- 2020}
		{Ph.D. International Fellowship}
		{National Agency for Research and Innovation (ANII)}
		{\textsterling 49,000 | Postgraduate Fellowship POS\_EXT\_2016\_1\_136663 (Lincoln, United Kingdom)}
	\entry
		{2017}
		{Travel Grant for Postgraduate Students}
		{Centro Brasileiro-Argentino de Biotecnologia (CBAB/CABBIO)}
		{\textsterling 1,800 | Attend the Course ‘Bioinformatics Tools for RNA-Sequence Data Analysis’ at Laboratório Nacional de Computação Científica (Petrópolis, RJ, Brasil)}
	\entry
		{2015}
		{Internship Grant for Postgraduate Students}
		{Association of Universities of the Montevideo Group (AUGM)}
		{\textsterling 1,500 | Internship stay at the Laboratório de Biodados, Universidade Federal de Minas Gerais (UFMG) (Belo Horizonte, Brazil)}
	\entry
		{2013}
		{Internship Grant for Postgraduate Students}
		{Program for the Development of Basic Sciences (PEDECIBA)}
		{\textsterling 1,500 | Internship stay at Bioinformatics at the Department of Biochemistry and Molecular Biology, University of Mississippi (Starkville, United States)}
	\entry
		{2013 -- 2015}
		{Posgraduate National Fellowship}
		{National Agency for Research and Innovation (ANII)}
		{\textsterling 8,500 | Postgraduate Fellowship POS\_NAC\_2012\_1\_8976 (Montevideo, Uruguay)}
	\entry
		{2012}
		{Travel Grant}
		{Latin American Network for Conservation Genetics (REGENEC)}
		{\textsterling 1,500 | Attend the ‘VIII Genetic Workshop for Conservation: new tools and new concepts’ (Chillán, Chile)}
	\entry
		{2012}
		{Mobility Grant}
		{Association of Universities of the Montevideo Group (AUGM)}
		{\textsterling 2,000 | A semester at Universidade Estadual de Jaboticabal (UNESP) (Jaboticabal, SP, Brazil)}
	\entry
		{2011 -- 2012}
		{Undergraduate National Fellowship}
		{National Agency for Research and Innovation (ANII)}
		{\textsterling 2,000 | Initiation in Research Fellowship BE\_INI\_2010\_1961 (Montevideo, Uruguay)}

\end{entrylist}

%----------------------------------------------------------------------------------------
%	STUDENTS
%----------------------------------------------------------------------------------------

\cvsect{Students}

\begin{description}
\item[]{\bf Enzo Cavalli}, BSc (Honours) in Biological Sciences, School of Science, {\bf Universidad de la Rep\'{u}blica}. Project: 'Effects of livestock on the mammal community of Paso Centurión, Cerro Largo'. Advisor, 2018-2019.  [\href{https://www.colibri.udelar.edu.uy/jspui/bitstream/20.500.12008/23484/6/uy24-19703.pdf}{thesis}]
\item[]{\bf Rodrigo Montiel}, BSc (Honours) in Biological Sciences, School of Science, {\bf Universidad de la Rep\'{u}blica}. Project: 'NaturalistaUY a citizen science tool to improve biodiversity knowledge in Uruguay'. Advisor, 2022-2023. [\href{https://www.colibri.udelar.edu.uy/jspui/bitstream/20.500.12008/42167/1/uy24-20995.pdf}{thesis}] [\href{https://github.com/Rodrigo-Montiel/TesisNaturalistaUY}{github repo}]
\end{description}


%----------------------------------------------------------------------------------------
%	LECTURING
%----------------------------------------------------------------------------------------

\cvsect{Lecturing}

\begin{entrylist}
	\entry
		{2022 -- 2023\\\footnotesize{two editions}}
		{Spatial Ecology and Macroecology}
		{Faculty of Environmental Sciences, CZU (Czech Republic)}
		{Practical classes in the posgraduate course for 10 students. Taught together with François Leroy and led by Dr. Petr Keil [\href{https://petrkeil.github.io/courses/post/2022/09/09/Spatial_ecology.html}{syllabus}]}
	\entry
		{2017 -- 2019\\\footnotesize{two editions}}
		{Modern Mammals}
		{School of Science, Universidad de la Rep\'{u}blica (Uruguay)}
		{Lecture in the graduate course of Palaeontology, for 15-20 students. Invited by Dr. Richard Fari\~{n}a [\href{https://flograttarola.com/pdf/mamiferos_julana.pdf}{slides}]}
	\entry
		{2018}
		{Management and analysis of camera trap records using R}
		{Ministry of Environment (Uruguay)}
		{Course for NGOs, researchers and government people. Taught together with Juan M. Barreneche and Gabriel Perazza [\href{https://github.com/bienflorencia/curso_camtrapR}{syllabus}]}
	\entry
		{2016 -- 2017\\\footnotesize{two editions}}
		{Introduction to Python}
		{Universidad de la Rep\'{u}blica, Uruguay}
		{Lecture at posgraduate course 'Introduction to command line and programming for bioinformatic analysis', for 25-30 students in Biological Sciences. Invited by Dr. Andr\'{e}s Iriarte [\href{https://github.com/bienflorencia/clases_python}{syllabus}]}
	\entry
		{2013 -- 2017\\\footnotesize{four editions}}
		{Relationships between Society and Nature}
		{Universidad de la Rep\'{u}blica, Uruguay}
		{Graduate course for 10-20 students of Anthropology, Biology, Geography and Social Sciences. Taught together with Carlos Santos, Javier Taks, Magdalena Chouhy, Andrea Garay, Luc\'{i}a Berg\'{o}s and Gabriel Perazza [\href{https://udelar.edu.uy/retema/actividades/grupos-de-trabajo/}{link to S\&N Group}]}
	\entry
		{2014}
		{Environmental education and community involvement}
		{Universidad de la Rep\'{u}blica, Uruguay}
		{Lecture at posgraduate course 'Conservation biology' for 15-20 students in Biological Sciences. Invited by Dr. Susana Gonz\'{a}lez. }
	\entry
		{2014}
		{Introduction to Biology}
		{Universidad de la Rep\'{u}blica, Uruguay}
		{Discussion group for the course Introduction to Biology School or 15-20 undergraduate students of Biological Sciences. Taught together with Dr. Juan Cristina.}
\end{entrylist}

%----------------------------------------------------------------------------------------
%	ACADEMIC MEMBERSHIPS
%----------------------------------------------------------------------------------------

\begin{minipage}[t]{0.35\textwidth}
	\vspace{-\baselineskip} % Required for vertically aligning minipages
	\cvsect{Service for Journals}
	
	\textbf{Frontiers of Biogeography}\\
	\textbf{Biodiversity Data Journal}\\
	\textbf{Global Ecology and Biogeography}\\
	\textbf{Journal of Biogeography}\\
	\textbf{Global Change Biology}

	\cvsect{Citation Metrics}

	\textbf{\href{https://scholar.google.com/citations?user=9KCM81IAAAAJ&hl}{Google Scholar}} (03/2024)\\
	218 citations, h-index = 8\\
\end{minipage} 
\hfill
\begin{minipage}[t]{0.65\textwidth}
	\vspace{-\baselineskip} % Required for vertically aligning minipages

	\cvsect{Academic Memberships}

	\textbf{2023} -- Ecological Society of America.\\
	\textbf{2021} -- International Biogeography Society.\\
	\textbf{2020} -- GBIF Biodiversity Open Data Ambassador.\\
	\textbf{2019} -- Organization for Women in Science for the Developing World (OWSD).\\
	\textbf{2018} -- British Ecological Society (BES).\\ 
	\textbf{2018} -- Research Data Alliance (RDA) - SHARC Interest Group.\\ 
	\textbf{2013} -- Interdisciplinary Group on Society-Nature Relationships (RETEMA).\\
	\textbf{2012} -- Latin American Network for Conservation Genetics (REGENEC).\\
	\textbf{2010} -- Zoological Society of Uruguay (SZU).\\\\
	
\end{minipage} 


%----------------------------------------------------------------------------------------
%	TRAINING
%----------------------------------------------------------------------------------------

\cvsect{TRAINING}

\begin{entrylist}
	\entrylong
		{2024}
		{ALTa Ciencia Abierta}
		{Virtual}
		{Metadocencia}
	\entrylong
		{2024}
		{Fitting Integrated Distribution Models with PointedSDMs}
		{Swansea, UK}
		{International Statistical Ecology Conference}
	\entrylong
		{2023}
		{Instructor Training: Data Carpentry, Library Carpentry, and Software Carpentry}
		{Virtual}
		{The Carpentries}
	\entrylong
		{2023}
		{NIMBLE course}
		{Virtual}
		{University of California, Berkeley}
	\entrylong
		{2022}
		{Joint Species Distribution Modelling with Hierarchical Modelling of Species Communities}
		{Jyväskylä, Finland}
		{Jyväskylä University}	
	\entrylong
		{2022}
		{Research Data Science Summer School}
		{Trieste, Italy}
		{CODATA-RDA}	
	\entrylong
		{2022}
		{Advanced Training Workshop on Science Communication}
		{Virtual}
		{Bardo Científico \& Academia de Ciencias de América Latina}	
	\entrylong
		{2020}
		{Using Python to Access Web Data}
		{Virtual}
		{Coursera}	
	\entrylong
		{2019}
		{Species Distribution Modelling: Fundamentals and the Future}
		{Nottingham, UK}
		{University of Nottingham and British Ecological Society}	
	\entrylong
		{2019}
		{Advancing in statistical modelling for evolutionary biologists and ecologists using R}
		{Glasgow, Scotland}
		{PR Statistics}	
	\entrylong
		{2017}
		{Bioinformatics Tools for RNA-Sequence Data Analysis}
		{Petrópolis, Brazil}
		{Laboratório Nacional de Computação Científica}
	\entrylong
		{2017}
		{From Complex Networks to Social Networks: Introduction to the Use of Big Data}
		{Montevideo, Uruguay}
		{Espacio Interdisciplinario de la Universidad de la República}
	\entrylong
		{2016}
		{WGCAC Human and Vertebrate Genomics: Bioinformatics Tools and Resources}
		{Montevideo, Uruguay}
		{Wellcome Trust UK}
	\entrylong
		{2015}
		{Next Generation Sequencing Strtegies}
		{Belo Horizonte, Brazil}
		{Universidad Federal de Minas Gerais}

\end{entrylist}

%----------------------------------------------------------------------------------------
%	TALKS
%----------------------------------------------------------------------------------------

\cvsect{Conference Talks}

\begin{entrylist}
	\entrylong
		{2024}
		{The scale-dependency of species associations in the eyes of integrated species distribution models}
		{Swansea, UK}
		{\underline{Grattarola F.}*, Guillera-Arroita G., Lahoz-Monfort J. \& Keil P. \slashsep International Statistical Ecology Conference [\href{https://flograttarola.com/talk/the-scale-dependency-of-species-associations-in-the-eyes-of-integrated-species-distribution-models/ISEC_2024_Grattarola_IJSDMs.pdf}{slides}]}
	\entrylong
		{2024}
		{NaturalistaUY un caso de ciencia comunitaria desde una perspectiva crítica}
		{Online}
		{\underline{Grattarola F.}* Bergós L., Carabio M., González S. \& Montiel R. \slashsep Conference for Advancing the Participatory Sciences [\href{https://bienflorencia.github.io/CAPS_2024_NaturalistaUY}{poster}]}
	\entrylong
		{2023}
		{Mesa redonda: Datos abiertos de biodiversidad en Uruguay}
		{Montevideo, Uruguay}
		{\underline{Grattarola F.}* \slashsep IV Congreso Uruguayo de Zoología [\href{https://flograttarola.com/talk/datos-abiertos-de-biodiversidad-en-uruguay/Mesa_redonda_datos_abiertosCUZ2023.pdf}{slides}] [\href{https://youtu.be/3hzSsP84u88}{video}]}
	\entrylong
		{2023}
		{Modelos integrados de distribución de especies para evaluar la dinámica en el rango de distribución geográfica de carnívoros Neotropicales}
		{Montevideo, Uruguay}
		{\underline{Grattarola F.}*, Bowler E. D. \& Keil P. \slashsep IV Congreso Uruguayo de Zoología}
	\entrylong
		{2023}
		{Estatus de la invasión de Carpobrotus edulis en Uruguay, basado en registros de ciencia ciudadana}
		{Montevideo, Uruguay}
		{\underline{Grattarola F.}*, Rodríguez-Tricot L., Zarucki M. \& Laufer G. \slashsep IV Congreso Uruguayo de Zoología [\href{https://flograttarola.com/talk/estatus-de-la-invasion-de-carpobrotus-edulis-en-uruguay-basado-en-registros-de-ciencia-ciudadana/Carpobrotus_edulis_CUZ2023.pdf}{slides}]}
		\entrylong
		{2023\\\footnotesize{Invited}}
		{The open-data tale of the yaguarundí}
		{Online}
		{\underline{Grattarola F.}* \slashsep Simposium - Datathon Latin America 2023 [\href{https://flograttarola.com/talk/el-cuento-del-yaguarundi-segun-los-datos-abiertos/simposio_DatatonLatam_2023.pdf}{slides}]}
		\entrylong
		{2023}
		{Assessing hotspots of geographic range changes for charismatic carnivores in the Neotropics using Integrated Species Distribution Models (ISDM)}
		{Portland, USA}
		{\underline{Grattarola F.}*, Bowler E. D. \& Keil P. \slashsep Ecological Society of America Annual Meeting (ESA) [\href{https://esa2023.eventscribe.net/fsPopup.asp?PresentationID=1276990&query=grattarola&Mode=presInfo}{abstract}]}
	\entrylong
		{2023}
		{Towards implementable recommendations (SHARC)}
		{Gothenburg, Sweden}
		{Shmagun H., \underline{Grattarola F.}*, Erdmann C., Cambon-Thomsen A., Thomsen M.  \& Mabile L. \slashsep 20th Plenary of the Research Data Alliance (RDA)  [\href{https://docs.google.com/document/d/1oQRp8EjHEOrmLj5QlqANM31q-AQa-CvJ804TRc9Re6I/edit?usp=sharing}{session notes}]}
	\entrylong
		{2022}
		{Data integration to model species geographic range dynamics. The yaguarundí (Herpailurus yagouaroundi) in Latin America as a case study}
		{Edinburgh, Scotland}
		{\underline{Grattarola F.}*, Bowler E. D. \& Keil P. \slashsep British Ecological Society Annual Meeting 2022 [\href{https://flograttarola.com/pdf/Grattarola_BES2022_short.pdf}{slides}]}
	\entrylong
		{2022\\\footnotesize{Invited}}
		{Biodiversidata: el Consorcio de Datos de Biodiversidad}
		{Sevilla, España}
		{\underline{Grattarola F.}* \slashsep Jornada sobre Estrategia Latino Americana, Caribeña e Iberoamericana para la Información de Biodiversidad (GBIF, LifeWatch ERIC) [\href{https://flograttarola.com/pdf/Biodiversidata_Sevilla.pdf}{slides}]}
	\entrylong
		{2022\\\footnotesize{Invited}}
		{Open data, community science and environmental education in Uruguay}
		{Rosario, Argentina - Online}
		{\underline{Grattarola F.}* \slashsep I Encuentro Internacional sobre Educación Abierta Ambiental: enseñanza, activismo e investigación [\href{https://flograttarola.com/pdf/IEEAA_2022.pdf}{slides}]}
	\entrylong
		{2022}
		{How Open Science activities are perceived and recognised in Research and the Research career}
		{Seoul, Korea}
		{\underline{Grattarola F.}*, Shmagun H., Mabile L. \& Cambon-Thomsen A. \slashsep International Data Week 2022 - RDA Plenary Breakout [\href{https://www.idw2022.org}{conference}]}
	\entrylong
		{2021}
		{As a naturalist in my country, where are there more opportunities to fill in information gaps?}
		{Online}
		{\underline{Grattarola F.}* \& Barreneche J.M. \slashsep LatinR Conference [\href{https://github.com/bienflorencia/LatinR2021}{code repo}]}
	\entrylong
		{2021}
		{Roundtable: Activism through digital media}
		{Montevideo, Uruguay}
		{\underline{Grattarola F.}* \slashsep 4th Conference on Digital Citizenship [\href{https://youtu.be/NNA46QEZVQI}{video}]}
	\entrylong
		{2021}
		{The value of open education policies to promote environmental education}
		{Online}
		{\underline{Grattarola F.}* \slashsep CILAC Latin American Open Science Forum [\href{https://youtu.be/W6i6y87p2IM}{video}]}
	\entrylong
		{2019}
		{Biodiversity hotspots in Uruguay: real or fabricated?}
		{Belfast, Northern Ireland}
		{\underline{Grattarola F.}* \& D. Pincheira-Donoso \slashsep British Ecological Society Annual Meeting [\href{https://flograttarola.com/pdf/FGrattarola_BES2019_WEB.pdf}{slides}]}
	\entrylong
		{2019}
		{Biodiversidata: a collaborative initiative towards open data availability in Uruguay}
		{Leiden, The Netherlands}
		{\underline{Grattarola F.}* \& D. Pincheira-Donoso \slashsep Biodiversity Next Conference [\href{https://flograttarola.com/pdf/BISS_article_37715.pdf}{abstract}]}
	\entrylong
		{2019\\\footnotesize{Invited}}
		{Biodiversity data in Uruguay: a long way to go}
		{Leiden, The Netherlands}
		{\underline{Grattarola F.}* \slashsep Biodiversity Informatics 101 Workshop at Biodiversity Next Conference}
	\entrylong
		{2019\\\footnotesize{Invited}}
		{Biodiversity hotspots in Uruguay}
		{Lincoln, United Kingdom}
		{\underline{Grattarola F.}* \& D. Pincheira-Donoso \slashsep School of Life Sciences Postgraduate Symposium, University of Lincoln}
	\entrylong
		{2019}
		{iNaturalist as a tool to enhance the conservation of biodiversity in Uruguay}
		{Maldonado, Uruguay (online)}
		{\underline{Grattarola F.}* \slashsep Rufford Conference Uruguay  [\href{https://youtu.be/3_lM-gPNLWY}{video}]}
	\entrylong
		{2017}
		{Landscape genetics of the crab-eating fox Cerdocyon thous (Carnivora: Canidae)}
		{Montevideo, Uruguay}
		{\underline{Grattarola F.}*, Bartesaghi L., Hernández D., Soutullo A., Mannise N., González S. \& Cosse M. \slashsep IV Uruguayan Zoology Congress}
	\entrylong
		{2016}
		{Humans and fauna in Paso Centurión (Cerro Largo) from an interdisciplinary perspective}
		{La Paloma, Uruguay}
		{Chouhy M.*, Garay A., Bergós L.,Gaucher L., Perazza G., \underline{Grattarola F.} \& Santos C. \slashsep III Interdisciplinary Conference on Biodiversity and Ecology}
	\entrylong
		{2016}
		{Democratization of information in the context of territorial management}
		{Montevideo, Uruguay}
		{Gaucher L.*, Hernández D., \underline{Grattarola F.}, Rodriguez-Tricot L., Duarte A., Perazza G. \& González, S \slashsep III Interdisciplinary Conference on Biodiversity and Ecology}
	\entrylong
		{2016}
		{Molecular detection of species and sex determination in road killed foxes in Uruguay}
		{Lavras, Brazil}
		{Mannise N.*, Cosse M., \underline{Grattarola F.}, Repetto L. \& González S. \slashsep First Ibero-American Congress on Biodiversity and Road Infrastructure}		
\end{entrylist}



%----------------------------------------------------------------------------------------
%	SCI COMM
%---------------------------------------------------------------------------------------- \hfill 

\cvsect{Sci Comm}

\begin{entrylist}
	\entrylong
		{2024}
		{The wonders of iNaturalist}
		{}
		{IIBCE Ecology and Evolutionary Biology Department [\href{https://flograttarola.com/talk/las-maravillas-de-inaturalist/iNat_DEB-IIBCE_seminario.pdf}{slides}]}
	\entrylong
		{2024}
		{First steps towards your Open Science journey}
		{}
		{Presentation at the IAVS EcoInformatics Seminar Series [\href{https://flograttarola.com/talk/first-steps-towards-your-open-science-journey/IAVS-EcoInfo_OpenScience_Flo.pdf}{slides}]}
	\entrylong
		{2023}
		{Uruguay nomá}
		{}
		{TV interview for the program Ciudad Viva of Tv Ciudad [\href{https://youtu.be/Edfpt810rt0}{video}]}
	\entrylong
		{2023}
		{ISDMs to assess the range dynamics of Neotropical carnivores}
		{}
		{Presentation at the seminar series Spatial Analysis of Biodiversity [\href{https://flograttarola.com/Modelado_de_NE_y_DG.pdf}{slides}] [\href{https://www.youtube.com/live/F0BpYC3iE4g}{video}]}
	\entrylong
		{2023}
		{Ciencia participativa: Naturalista Uruguay}
		{}
		{Presentation at the Foro Mujeres NaturaListas en México y Latinoamérica [\href{https://flograttarola.com/talk/ciencia-participativa-y-naturalista-uruguay/Foro_mujeres_Mexico_y_Latam.pdf}{slides}] [\href{https://fb.watch/iBdjBHYFQm/}{video}]}
	\entrylong
		{2023}
		{iNaturalist the app with which anyone can provide information on flora and fauna in Uruguay}
		{}
		{Radio interview by Gastón González for En Perspectiva, Radiomundo 1170 AM [\href{https://youtu.be/TcYk9ZquyvE}{video}] [\href{https://soundcloud.com/en-perspectiva-uy/entrevista-florencia-grattarola-naturalistauy-te-invita-a-realizar-ciencia-ciudadana}{soundcloud}]}
	\entrylong
		{2022}
		{Workshop in preparation for the Greatern Southern Bioblitz 2022 in NaturalistaUY}
		{}
		{Live talk streamed on YouTube for NaturalistaUY users [\href{https://youtu.be/sPKhUy3CJQ0}{video}]}
	\entrylong
		{2022}
		{Community science outreach with JULANA and NaturalistaUY}
		{}
		{Radio interview by Diego Varela and Fernanda Cabrera for Radio Fénix [\href{https://youtu.be/Scc01vg1PTs}{video}]}
	\entrylong
		{2022}
		{Scientific Pokemon Go}
		{}
		{Podcast episode by Soledad Machado for Medios Públicos [\href{https://open.spotify.com/episode/05X1u40HyvvTjuuYKjyJE2?si=jCNZ8hHmQ8S-CSfiNOCryQ}{episode}]}
	\entrylong
		{2020}
		{Biodiversidata began to digitize the herbarium of the National Museum of Natural History}
		{}
		{Press note by Leo Lagos for La Diaria [\href{https://ladiaria.com.uy/ciencia/articulo/2020/2/biodiversidata-comenzo-a-digitalizar-el-herbario-del-museo-nacional-de-historia-natural/}{article}]}
	\entrylong
		{2021}
		{iNaturalist - What is it and what can it do for our project?}
		{}
		{Talk with students of the Polo Educativo Tecnológico Arrayanes (Park Rangers) [\href{https://flograttarola.com/pdf/Arrayanes_iNatUY.pdf}{slides}]}
	\entrylong
		{2020}
		{Getting started with iNaturalist - Project RACATÚ}
		{}
		{Online talk with primary school students from rural school N102 of Berachi (Cerro Largo, Uruguay) [\href{https://flograttarola.com/pdf/iNat_Berachi.pdf}{slides}]}
	\entrylong
		{2020}
		{Uruguayans are wanted to participate in a platform that helps scientists}
		{}
		{Press note by Analía Filosi for Diario El País [\href{https://www.elpais.com.uy/vida-actual/busca-uruguayos-participen-plataforma-ayuda-cientificos.html}{press article}]}
	\entrylong
		{2019}
		{iNaturalist: a bet on citizen science}
		{}
		{Talk at Museum of Natural History Torres de la Llosa [\href{https://flograttarola.com/pdf/NaturalistaUy.pdf}{slides}]}
	\entrylong
		{2019}
		{iNaturalist a citizen science tool to improve biodiversity knowledge in Uruguay}
		{}
		{Radio interview by Juan Vique in Radio Uruguay [\href{https://sobreciencia.uy/inaturalist-una-apuesta-a-la-ciencia-ciudadana/}{audio clip}]}
	\entrylong
		{2019}
		{The Uruguayan Consortium of Biodiversity Data, Biodiversidata, published the first open data set}
		{}
		{Press note by Leo Lagos for La Diaria [\href{https://ladiaria.com.uy/ciencia/articulo/2019/7/el-consorcio-de-datos-de-biodiversidad-de-uruguay-biodiversidata-publico-el-primer-set-de-datos-abiertos/}{press article}]}
	\entrylong
		{2019}
		{Series 'Gender and IT' -  Pan Ceibal}
		{}
		{Online talk with students from Canelones, Paysandú, Tacuarembó and Treinta y Tres' primary schools in Uruguay}
	\entrylong
		{2019}
		{Biodiversidata: Biodiversity Open Data}
		{}
		{Online talk at the Third Edition of 'Datos y Cogollos: Open Data in Uruguay'. La Diaria Media Lab (Montevideo, Uruguay) [\href{https://flograttarola.com/talk/biodiversidata.-datos-abiertos-de-biodiversidad/Biodiversidata_Datos_y_Cogollos_OCT2019.pdf}{slides}]}
	\entrylong
		{2019}
		{Exhibition ‘A Computer of One’s Own’}
		{}
		{Art exhibition about pioneer women in Computer Science at CSSconf and JSConf (Berlin, Germany) [\href{https://medium.com/a-computer-of-ones-own}{project's website}]}
	\entrylong
		{2018}
		{Big data in biodiversity}
		{}
		{Radio interview with Federico Weinstein in the science segment of the program Justicia Infinita, Radio Océano (93.9 FM) (Montevideo, Uruguay)}
	\entrylong
		{2017}
		{The Marsh Seedeater}
		{}
		{Video production for PROBIDES organization [\href{https://youtu.be/EYKt83ShWQ8}{video}]}
	\entrylong
		{2016}
		{Drawn news: First record of a jaguarundi in Uruguay}
		{}
		{Video production for JULANA NGO [\href{https://youtu.be/Zva9m9hmXCc}{video}]}	
\end{entrylist}



%----------------------------------------------------------------------------------------
%	ADDITIONAL INFORMATION
%----------------------------------------------------------------------------------------

%\begin{minipage}[t]{0.3\textwidth}
%	\vspace{-\baselineskip} % Required for vertically aligning minipages
%
%	\cvsect{Languages}
%	
%	\textbf{English} - native\\
%	\textbf{German} - proficient\\
%	\textbf{Polish} - rudimentary
%\end{minipage}
%\hfill
%\begin{minipage}[t]{0.3\textwidth}
%	\vspace{-\baselineskip} % Required for vertically aligning minipages
%	
%	\cvsect{Hobbies}
%	
%	I love... \lorem
%\end{minipage}
%\hfill
%\begin{minipage}[t]{0.3\textwidth}
%	\vspace{-\baselineskip} % Required for vertically aligning minipages
%	
%	\cvsect{Non profit}
%	
%	I help... \lorem
%\end{minipage}

%----------------------------------------------------------------------------------------
%\cvsect{References}
%
%\begin{entrylist}
%	\entryshort
%		{Petr Keil}
%		{Czech University of Life Sciences, Czech Republic}
%		{}
%		{\href{mailto:keil@fzp.czu.cz}{keil@fzp.czu.cz} | phone +420 776 371 840}
%	\entryshort
%		{Daniel Pincheira-Donoso}
%		{Queen’s University Belfast, UK}
%		{}
%		{\href{mailto:d.pincheira-donoso@qub.ac.uk}{d.pincheira-donoso@qub.ac.uk} | phone +44(0) 28 9097 2967}
%	\entryshort
%		{Lauren Weatherdon}
%		{UN Environment Programme World Conservation Monitoring Centre, UK}
%		{}
%		{\href{mailto:lauren.weatherdon@unep-wcmc.org}{lauren.weatherdon@unep-wcmc.org} | phone +420 776 371 840}
%	\entryshort
%		{Susana González}
%		{Clemente Estable Biological Research Institute, Uruguay}
%		{}
%		{\href{mailto:sugonza9@yahoo.com}{sugonza9@yahoo.com} | phone +598 24861417}		
%\end{entrylist}
%
%
%
%
%\begin{center}
%	\bubbles{1/, 2/, 3/, 4/, 3/, 2/, 1/}
%\end{center}

\end{document}
